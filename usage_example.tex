\documentclass[a4paper, 12pt]{extarticle}

% Enter all fields in the following and run the compilation TWICE [!]
\newcommand{\docType}{AFF XXX}
\newcommand{\client}{CLIENT}
\newcommand{\projectName}{Contrôle avec des caméras}
\newcommand{\documentName}{Essais du template \LaTeX}
\newcommand{\redactor}{J. LEDIG}
\newcommand{\titlePageLogo}{title_page_logo.png}

% You can remove that package, but add needed packages here, before using the \input command!
\usepackage{blindtext}
% Used packages
\usepackage{xcolor}
\usepackage{fancyhdr}
\usepackage[hidelinks]{hyperref}
\usepackage{graphicx}
\usepackage{amsmath}
\usepackage{setspace}
\onehalfspacing

% The font
\usepackage[T1]{fontenc}
\usepackage[utf8]{inputenc}
\usepackage{palatino}
\renewcommand*\familydefault{\sfdefault}
\usepackage[cm]{sfmath}

% Python code
\usepackage{listings}
\definecolor{codegreen}{rgb}{0,0.6,0}
\definecolor{codegray}{rgb}{0.5,0.5,0.5}
\definecolor{codepurple}{rgb}{0.58,0,0.82}
\definecolor{backcolour}{rgb}{0.95,0.95,0.92}
 
\lstdefinestyle{py}{
    backgroundcolor=\color{backcolour},   
    commentstyle=\color{codegreen},
    keywordstyle=\color{magenta},
    numberstyle=\tiny\color{codegray},
    stringstyle=\color{codepurple},
    basicstyle=\footnotesize,
    breakatwhitespace=false,         
    breaklines=true,                 
    captionpos=b,                    
    keepspaces=true,                 
    numbers=left,                    
    numbersep=5pt,                  
    showspaces=false,                
    showstringspaces=false,
    showtabs=false,                  
    tabsize=2
}\definecolor{codegreen}{rgb}{0,0.6,0}
\definecolor{codegray}{rgb}{0.5,0.5,0.5}
\definecolor{codepurple}{rgb}{0.58,0,0.82}
\definecolor{backcolour}{rgb}{0.95,0.95,0.92}
 
\lstdefinestyle{py}{
    backgroundcolor=\color{backcolour},   
    commentstyle=\color{codegreen},
    keywordstyle=\color{magenta},
    numberstyle=\tiny\color{codegray},
    stringstyle=\color{codepurple},
    basicstyle=\footnotesize,
    breakatwhitespace=false,         
    breaklines=true,                 
    captionpos=b,                    
    keepspaces=true,                 
    numbers=left,                    
    numbersep=5pt,                  
    showspaces=false,                
    showstringspaces=false,
    showtabs=false,                  
    tabsize=2
}
\lstset{style=py}
\lstset{language=Python}
\renewcommand{\lstlistingname}{Script Python}

% Tables
\usepackage{booktabs}
\usepackage{multirow}

% The french
\usepackage[french]{babel}
\usepackage{subcaption}
\DeclareCaptionFormat{custom}
{%
    \textbf{#1#2}\textit{\small #3}
}
\captionsetup{format=custom}
\captionsetup[table]{name=Tableau}
\captionsetup{labelfont={bf},format=hang,labelsep=period}



% The margin
\usepackage[left=2.5cm, right=2.5cm, top=2.4cm, bottom=2.5cm]{geometry}

% The background layout
\usepackage[pages=1]{background}
\backgroundsetup{scale = 1.01, angle = 0, opacity = 1,
  contents = {
  			\ifnum\thepage = 1 \includegraphics[width = \paperwidth, height = \paperheight, keepaspectratio]{AXDoc/bg_title.pdf}
  			\else \includegraphics[width = \paperwidth, height = \paperheight, keepaspectratio]{AXDoc/bg_all.pdf}
  			\fi}}

% Header and footer
\pagestyle{fancy}
\renewcommand\headrulewidth{0pt}
\fancyhf{} 
\fancyfootoffset[R]{-0.35cm}
\fancyfoot[R]{\color{white}\vspace{-0.05cm}\textbf{\thepage}}
\fancyheadoffset[R]{-0.45cm}
\fancyhead[R]{\documentName \hspace{0.7cm}\textbf{\color{white}\docType}}


\begin{document}
% The title page
\thispagestyle{empty}
\begin{center}
       \vspace*{0.15\textheight}
       \textbf{\huge {\docType~--~\client}}
       
       \vspace{0.8cm}
       \textbf{\huge \projectName}   
       
       \vspace{0.8cm}
       \includegraphics[width=0.4\textwidth]{\titlePageLogo}
       \vspace{0.8cm}
       
       
       \textbf{\huge \documentName}
       \vfill
       
       
       Ce document est strictement confidentiel et à destination exclusive du client,\\ \client .
       
       \vspace{0.8cm}
       
       
       Date de la dernière mise à jour du document : \today \\
       Rédacteur(s) du document : \redactor
            
\end{center}
\newpage
\tableofcontents
\newpage



\section{Une section}
\subsection{Une sous section}
\blindtext[1]
\subsection{Ajout d'une image}
Voyons comment ajouter une image, la figure \ref{img:Une_image} dans ce template :
\begin{figure}[hb]
\centering
\includegraphics[width=0.5\textwidth]{title_page_logo.png}
\caption{La légende}\label{img:Une_image}
\end{figure}

\subsection{Ajout d'une équation}
Même si nous n'utilisons pas souvent des équations complexes, il est important de savoir comment les intégrer dans un document, 
\begin{equation}\label{eq:equation_einstein}
G_{\mu\nu}=\frac{8\pi G}{c^4}T_{\mu\nu}
\end{equation}
Ici, \eqref{eq:equation_einstein} est l'équation d'Einstein. Le terme de gauche est le tenseur métrique définissant la courbure de l'espace-temps en fonction du tenseur énergie-masse. 

\subsection{Ajout d'un morceau de code}
On peut entrer du code avec la commande \texttt{lstlinsting}, voir code \ref{code:mon_code}. 
\begin{lstlisting}[caption={Use logging}, label={code:mon_code}]
import os
import numpy as np

% Use logging because la camera chauffe
print("logging")
\end{lstlisting}

\subsection{Ajout d'un tableau}
On peut également écrire un tableau, d'après le tableau \ref{tab:mon_tableau} : 
\begin{table}[h]
    \centering
    \begin{tabular}{lllll}
        \toprule
        \multirow{2}{*}{Models} & \multicolumn{3}{c}{Metric 1} & Metric 2\\
        \cmidrule{2-4} \cmidrule{5-5} \\
        {} & precision & recall & F-score  & R@10 \\
        \midrule
        model 1 & 0.67  & 0.8 & 0.729  & 0.75 \\
        model 2 & 0.8 & 0.9 & 0.847 & 0.85 \\
        \bottomrule
    \end{tabular}
    \caption{Un tableau avec des données très importantes !}\label{tab:mon_tableau}
\end{table}

\section{Comment utiliser ce template ?}
C'est assez simple : 
\begin{itemize}
\item Téléchargez le dossier "AXDoc",
\item à côté de ce doc, écrivez un fichier tex avec ces informations :
\begin{verbatim}
\documentclass[a4paper, 12pt]{extarticle}

% Enter all fields in the following and run the compilation TWICE [!]
\newcommand{\docType}{AFF XXX}
\newcommand{\client}{CLIENT}
\newcommand{\projectName}{Contrôle avec des caméras}
\newcommand{\documentName}{Essais du template \LaTeX}
\newcommand{\redactor}{J. LEDIG}
\newcommand{\titlePageLogo}{title_page_logo.png}

% You can remove that package, but add needed packages here, before using the \input command!
\usepackage{blindtext}
% Used packages
\usepackage{xcolor}
\usepackage{fancyhdr}
\usepackage[hidelinks]{hyperref}
\usepackage{graphicx}
\usepackage{amsmath}
\usepackage{setspace}
\onehalfspacing

% The font
\usepackage[T1]{fontenc}
\usepackage[utf8]{inputenc}
\usepackage{palatino}
\renewcommand*\familydefault{\sfdefault}
\usepackage[cm]{sfmath}

% Python code
\usepackage{listings}
\definecolor{codegreen}{rgb}{0,0.6,0}
\definecolor{codegray}{rgb}{0.5,0.5,0.5}
\definecolor{codepurple}{rgb}{0.58,0,0.82}
\definecolor{backcolour}{rgb}{0.95,0.95,0.92}
 
\lstdefinestyle{py}{
    backgroundcolor=\color{backcolour},   
    commentstyle=\color{codegreen},
    keywordstyle=\color{magenta},
    numberstyle=\tiny\color{codegray},
    stringstyle=\color{codepurple},
    basicstyle=\footnotesize,
    breakatwhitespace=false,         
    breaklines=true,                 
    captionpos=b,                    
    keepspaces=true,                 
    numbers=left,                    
    numbersep=5pt,                  
    showspaces=false,                
    showstringspaces=false,
    showtabs=false,                  
    tabsize=2
}\definecolor{codegreen}{rgb}{0,0.6,0}
\definecolor{codegray}{rgb}{0.5,0.5,0.5}
\definecolor{codepurple}{rgb}{0.58,0,0.82}
\definecolor{backcolour}{rgb}{0.95,0.95,0.92}
 
\lstdefinestyle{py}{
    backgroundcolor=\color{backcolour},   
    commentstyle=\color{codegreen},
    keywordstyle=\color{magenta},
    numberstyle=\tiny\color{codegray},
    stringstyle=\color{codepurple},
    basicstyle=\footnotesize,
    breakatwhitespace=false,         
    breaklines=true,                 
    captionpos=b,                    
    keepspaces=true,                 
    numbers=left,                    
    numbersep=5pt,                  
    showspaces=false,                
    showstringspaces=false,
    showtabs=false,                  
    tabsize=2
}
\lstset{style=py}
\lstset{language=Python}
\renewcommand{\lstlistingname}{Script Python}

% Tables
\usepackage{booktabs}
\usepackage{multirow}

% The french
\usepackage[french]{babel}
\usepackage{subcaption}
\DeclareCaptionFormat{custom}
{%
    \textbf{#1#2}\textit{\small #3}
}
\captionsetup{format=custom}
\captionsetup[table]{name=Tableau}
\captionsetup{labelfont={bf},format=hang,labelsep=period}



% The margin
\usepackage[left=2.5cm, right=2.5cm, top=2.4cm, bottom=2.5cm]{geometry}

% The background layout
\usepackage[pages=1]{background}
\backgroundsetup{scale = 1.01, angle = 0, opacity = 1,
  contents = {
  			\ifnum\thepage = 1 \includegraphics[width = \paperwidth, height = \paperheight, keepaspectratio]{AXDoc/bg_title.pdf}
  			\else \includegraphics[width = \paperwidth, height = \paperheight, keepaspectratio]{AXDoc/bg_all.pdf}
  			\fi}}

% Header and footer
\pagestyle{fancy}
\renewcommand\headrulewidth{0pt}
\fancyhf{} 
\fancyfootoffset[R]{-0.35cm}
\fancyfoot[R]{\color{white}\vspace{-0.05cm}\textbf{\thepage}}
\fancyheadoffset[R]{-0.45cm}
\fancyhead[R]{\documentName \hspace{0.7cm}\textbf{\color{white}\docType}}


\begin{document}
% The title page
\thispagestyle{empty}
\begin{center}
       \vspace*{0.15\textheight}
       \textbf{\huge {\docType~--~\client}}
       
       \vspace{0.8cm}
       \textbf{\huge \projectName}   
       
       \vspace{0.8cm}
       \includegraphics[width=0.4\textwidth]{\titlePageLogo}
       \vspace{0.8cm}
       
       
       \textbf{\huge \documentName}
       \vfill
       
       
       Ce document est strictement confidentiel et à destination exclusive du client,\\ \client .
       
       \vspace{0.8cm}
       
       
       Date de la dernière mise à jour du document : \today \\
       Rédacteur(s) du document : \redactor
            
\end{center}
\newpage
\tableofcontents
\newpage
\end{verbatim}
\item Compilez 2 fois, et c'est parti !
\end{itemize}


\end{document}